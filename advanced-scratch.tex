\documentclass{article}
\usepackage[a4paper, hmargin={2.8cm, 2.8cm}, vmargin={2.5cm, 2.5cm}]{geometry}
%%%%%%%%%%%%%%%%%%%%%%%%%%%%%%%%%%%%%%%%%%%%%%%%%%%%%%%%%%%%%%%%%%%%%%%%%%%%%%%%

%%%%%%%%%%%%%%%%%%%%%%%%%%%%%%%%%%%%%%%%%%%%%%%%%%%%%%%%%%%%%%%%%%%%%%%%%%%%%%%%
\usepackage[utf8]{inputenc}
\usepackage[T1]{fontenc}
%%%%%%%%%%%%%%%%%%%%%%%%%%%%%%%%%%%%%%%%%%%%%%%%%%%%%%%%%%%%%%%%%%%%%%%%%%%%%%%%


%%%%%%%%%%%%%%%%%%%%%%%%%%%%%%%%%%%%%%%%%%%%%%%%%%%%%%%%%%%%%%%%%%%%%%%%%%%%%%%%
\usepackage{mathtools}
\usepackage{amsthm}
\usepackage{amssymb}
\usepackage{csvsimple}
\usepackage{subcaption}
\usepackage{url}
\usepackage{url}
\usepackage{tikz}
\usepackage{pgfplots}

%%%%%%%%%%%%%%%%%%%%%%%%%%%%%%%%%%%%%%%%%%%%%%%%%%%%%%%%%%%%%%%%%%%%%%%%%%%%%%%%


%%%%%%%%%%%%%%%%%%%%%%%%%%%%%%%%%%%%%%%%%%%%%%%%%%%%%%%%%%%%%%%%%%%%%%%%%%%%%%%%
\usepackage{fancyhdr}
\usepackage{graphicx}
\usepackage{parskip}
\usepackage{listings}
\usepackage{enumitem}
\usepackage{titlesec}
\usepackage[lastpage,user]{zref}
\usepackage{caption}
\usepackage{scrextend}
\usepackage[outputdir=./.latex-out]{minted} % TODO slet hvis du ikke bruger minted
\usepackage{listings}
\usepackage{blindtext}

\title{\LaTeX{} for øvede webinar}

\author{Benjamin Rotendahl}

\date{\today}

% Variables
\def\populationCount{1337}
\newcommand{\LogicRule}[3]{\frac{#1 \quad #2}{#3}}
\usepackage{chemfig}
\usepackage{semantic}
\begin{document}



\maketitle

\begin{abstract}
	Abstract goes here...
\end{abstract}

\section{Introduktion}
Som vi så i det første webinar, er \LaTeX{} et værktøj til at skrive dokumenter.
I dette webinar vil vi se på nogle af de mere avancerede funktioner i \LaTeX{}.
Vi har i alt \populationCount{} deltagere i dette webinar.

\section{Avanceret Matematik i \LaTeX}
vi kender inline math mode som \(x^2 + y^2 = z^2\) og display math mode som
\[
	\left (\frac{x}{2} + 1  \right )^2 + y ^2 = z ^2 +
\]
% Flalign osv.
% Vector matricer
% Text i matematik (cases)
% detexify
% parenteser
% Logik formularer



\section{variabler og makroer}
\begin{picture}(220,75)(0,-35)
	\put(110,0){\program{P,\compiler{C,\machine{M},\program{P,M}}}}
\end{picture}
% skriv din egen funktion
% forsøgs resultat som variabel
% Logik formulrer genbrug


\[
	\LogicRule{p => q }{\lnot q}{\lnot p}
\]

To define chemical formulae you can use units.

\section{Pakker}
\chemfig{*6((=O)-N(-H)-(*5(-N=-N(-H)-))=-(=O)-N(-H)-)}
% Tex live Utility
% Semantic
% Chemfig
% https://www.overleaf.com/learn/latex/Chemistry_formulae


\section{Citationer i \LaTeX}
% Bibtex som database
% Find citationer online
% Typer af citationer
% Forskellige stilarter
% https://www.overleaf.com/learn/latex/Bibtex_bibliography_styles#Table_of_stylename_values



\section{Automatisk data indlæsning}
% Tables først
% CSV
% Transponer osv.
\begin{table}[h]
	\centering\csvautotabular[]{assets/dices1.csv}
	\caption{Automatic table from csc}\label{table:dices}
\end{table}

\section{Kodetekst}
\begin{listing}[!h]
	\begin{lstlisting}[tabsize=1]
		def fib(n):
			# Fibonacci numbers
			if n < 2:
				return 1
			else :
				return fib(n-1) + fib(n-2)
		\end{lstlisting}
	\caption{Eksempel på python kide}\label{lst:par}
\end{listing}

\begin{listing}[!h]
	\begin{minted}[tabsize=1]{python}
		def fib(n):
			# Fibonacci numbers
			if n < 2:
				return 1
			else :
				return fib(n-1) + fib(n-2)
		\end{minted}
	\caption{Eksempel på python kide}\label{lst:par}
\end{listing}

\begin{listing}[!h]
	\inputminted[firstline=0, lastline=12]{python}{assets/generator.py}
	\caption{Eksempel på python kide}\label{lst:par}
\end{listing}

% https://ctan.org/pkg/minted?lang=en
% https://tikz.net
\section{Tikz}
\begin{figure}[h]
	\centering
	\begin{tikzpicture}
		\begin{axis}[xmin=-10,xmax=10,ymin=-20,ymax=20, samples=100]
			\addplot[blue, thick] (x,5*x^3+3*x);
			\addplot[red, thin] (x*x,x);
		\end{axis}
	\end{tikzpicture}
	\caption{Eksempel på en tikz figur}\label{fig:tikz}
\end{figure}


\begin{figure}[h]
	\centering
	\begin{tikzpicture}
		\coordinate (a) at (3,5);
		\coordinate (b) at (5,3);
		\draw[] (a) -- (b);

	\end{tikzpicture}
	\caption{Eksempel på en tikz figur}\label{fig:tikz}
\end{figure}

\section{Beamer}

% \tableofcontents
% Beamer?

\newpage
\appendix
\bibliographystyle{apalike}
\bibliography{assets/references}{}
% https://www.overleaf.com/learn/latex/Bibtex_bibliography_styles#Table_of_stylename_values


\end{document}

x