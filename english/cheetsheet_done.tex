%%%%%%%%%%%%%%%%%%%%%%%%%%%%%%%%%%%%%%%%%%%%%%%%%%%%%%%%%%%%%%%%%%%%%%%%%%%%%%%%
\documentclass{article}
\usepackage[a4paper, hmargin={2.8cm, 2.8cm}, vmargin={2.5cm, 2.5cm}]{geometry}
%%%%%%%%%%%%%%%%%%%%%%%%%%%%%%%%%%%%%%%%%%%%%%%%%%%%%%%%%%%%%%%%%%%%%%%%%%%%%%%%

%%%%%%%%%%%%%%%%%%%%%%%%%%%%%%%%%%%%%%%%%%%%%%%%%%%%%%%%%%%%%%%%%%%%%%%%%%%%%%%%
\usepackage[utf8]{inputenc}
\usepackage[T1]{fontenc}
%%%%%%%%%%%%%%%%%%%%%%%%%%%%%%%%%%%%%%%%%%%%%%%%%%%%%%%%%%%%%%%%%%%%%%%%%%%%%%%%


%%%%%%%%%%%%%%%%%%%%%%%%%%%%%%%%%%%%%%%%%%%%%%%%%%%%%%%%%%%%%%%%%%%%%%%%%%%%%%%%
\usepackage{mathtools}
\usepackage{amsthm}
\usepackage{amssymb}
\usepackage{csvsimple}
\usepackage{subcaption}
\usepackage{url}
\usepackage{tikz}\usepackage{pgfplots}

%%%%%%%%%%%%%%%%%%%%%%%%%%%%%%%%%%%%%%%%%%%%%%%%%%%%%%%%%%%%%%%%%%%%%%%%%%%%%%%%


%%%%%%%%%%%%%%%%%%%%%%%%%%%%%%%%%%%%%%%%%%%%%%%%%%%%%%%%%%%%%%%%%%%%%%%%%%%%%%%%
\usepackage{fancyhdr}
\usepackage{graphicx}
\usepackage{parskip}
\usepackage{listings}
\usepackage{enumitem}
\usepackage{titlesec}
\usepackage[lastpage,user]{zref}
\usepackage{caption}
\usepackage{scrextend}
\usepackage[outputdir=./.latex-out]{minted} % TODO remove if you don't use minted
\usepackage{listings}
\usepackage{blindtext}
%%%%%%%%%%%%%%%%%%%%%%%%%%%%%%%%%%%%%%%%%%%%%%%%%%%%%%%%%%%%%%%%%%%%%%%%%%%%%%%%
\newcommand{\python}[1] {
  \mintinline{python}{#1}
}
\newcommand{\tex}[1] {
  \mintinline{latex}{#1}
}
\pagestyle{fancy}
%%%%%%%%%%%%%%%%%%%%%%%%%%%%%%%%%%%%%%%%%%%%%%%%%%%%%%%%%%%%%%%%%%%%%%%%%%%%%%%%
%%%%%%%%%%%%%%%%%%%%%%%%%%%%%%%%%%%%%%%%%%%%%%%%%%%%%%%%%%%%%%%%%%%%%%%%%%%%%%%%
\lhead{\LaTeX webinar} % TODO insert left page header
\rhead{Study Now} % TODO insert right page header
\cfoot{\thepage\ of \zpageref{LastPage}}
\newtheorem*{prp}{Propostion}
\setlist{nolistsep}
%%%%%%%%%%%%%%%%%%%%%%%%%%%%%%%%%%%%%%%%%%%%%%%%%%%%%%%%%%%%%%%%%%%%%%%%%%%%

%%%%%%%%%%%%%%%%%%%%%%%%%%%%%%%%%%%%%%%%%%%%%%%%%%%%%%%%%%%%%%%%%%%%%%%%%%%%
\title{
  \vspace{13em}
  \large{Study Now} \\
  \Large{\LaTeX{} webinar} \\
}

\author{
  Benjamin Rotendahl --- Benjamin@Rotendahl.dk
}

\date{
  \vspace{22em}
  \today
}
%%%%%%%%%%%%%%%%%%%%%%%%%%%%%%%%%%%%%%%%%%%%%%%%%%%%%%%%%%%%%%%%%%%%%%%%%%%%
%%%%%%%%%%%%%%%%%%%%%%%%%%%%%%%%%%%%%%%%%%%%%%%%%%%%%%%%%%%%%%%%%%%%%%%%%%%%%%%%

\begin{document}

\clearpage

% The following lines creates the title, table of contents and the frontpage
%%%%%%%%%%%%%%%%%%%%%%%%%%%%%%%%%%%%%%%%%%%%%%%%%%%%%%%%%%%%%%%%%%%%%%%%%%%%%
\maketitle
\thispagestyle{empty}
\newpage

\thispagestyle{empty}\tableofcontents\newpage % TODO remove this line to remove table of contents

\setcounter{page}{1}
%%%%%%%%%%%%%%%%%%%%%%%%%%%%%%%%%%%%%%%%%%%%%%%%%%%%%%%%%%%%%%%%%%%%%%%%%%%%%

\section{Introduction}
 This document is as a \emph{cheat sheet}. It covers how to setup a \LaTeX{}
 document and showcases the most common \LaTeX{} commands and functions. You
 have both the pdf and source code such that you can add other cool commands
 you find on your journey.

\section{\LaTeX{} background}
 Before diving into the syntax of \LaTeX{} you should know a bit about the
 process of writing \LaTeX{} documents. \LaTeX{} is a typesetting program and
 language. The main advantage of \LaTeX{} is its ability to typeset math, code
 and other scientific/technical figures. \LaTeX{} in an extension of TeX, which
 released by Donald Knuth in the year 1981.

 \subsection{From source code to PDF}\label{sec:local}
   \LaTeX{} is not a word processor, but a language and compiler which given a
   file containing source code, written in an \emph{editor}, creates a PDF.

   An editor for local use, could be \emph{Visual studio code}\cite{vscode}, it
   can be used to write any programming/markup language. Support for languages
   comes through packages that extends its functionality, there exists packages
   for most languages such as python, javascript and \LaTeX{}\cite{latexPackage}.

   After the editor has saved the source code, it must be passed to the compiler.
   This can be setup to happen automatically on save. There are different versions
   of the compiler, they differ in the amount of packages and features they come
   with.  ``The latex project TeX''\cite{texLive} has links to various distributions
   for the most common operating systems.
   With an editor and compiler installed you are ready to write \LaTeX{}.
   The processes is illustrated in figure~\ref{fig:compile}.

   \subsubsection{Overleaf}
     Section~\ref{sec:local} explains how to create a \LaTeX{} setup on your own
     computer. There exists services such as Overleaf\footnote{\url{overleaf.com}},
     they are to \LaTeX{} what Google docs is to Microsoft Word. They live in the
     browser and give you a full environment, no installation required. This makes
     it easier to share documents with potential group members, but has the
     disadvantage that they require an internet connection to function. They are
     also more limited in the amount of configuration options and packages available.

     \begin{figure}[h]
       \centering\includegraphics[width=0.9\textwidth]{../assets/compile_en.png}
       \caption{From source code to pdf}\label{fig:compile}
     \end{figure}


\section{\LaTeX{} syntax}
 In \LaTeX{} we don't click buttons to change formatting, instead we tell the
 compiler what type of text we are writing. This leaves the appearance to our
 template/compiler leaving us, the writer, to focus on the content.

 A \LaTeX{} file consists of two parts, the \emph{preamble} and the document.
 The preamble is where we define the \LaTeX{} packages we are going to use,
 the style of our document, macros, and other configurations options.
 The document is where we write the content.

 \subsection{Command syntax}
   A command consists of a ``backslash'' followed by a command name. If the
   command takes parameters they are written after the command name in curly
   parenthesis. As an example if we want to write \LaTeX{} we type \tex{\LaTeX}.
   Below is a list of the most common syntax constructs.
   \begin{description}
     \item[New lines] \LaTeX{} handles newlines for us, if you write a single
       line break it will be ignored, so you can format your source code as you
       wish. If you want to force a line break you can write two line breaks, write
       \tex{\linebreak} or type two backslashes \tex{\\}.

     \item[Forced spaces] As with line breaks, more than one space are ignored
       To force a space you can write \tex{\space} or \tex{~~}.

     \item[New page] To force latex to start a new page we type \tex{\newpage}

     \item[Qoutes] To write quotes such as ``Some thing'', we type plings and
       apostrophes: \tex{``qouted text''}.

     \item[Comments]
       If you want to type something in your source code but not include it in the
       final pdf, we can use comments. This is a useful way to leave notes to your
       future self or group mates. typing \(\%\) will result latex in ignored
       the remainder of the line. To type  \(\%\), prefix it with a backslash \tex{\%}.


     \item[Math] There are two ways to type math in \LaTeX{}. The first is to
       \emph{inline}, which places the expression in the flow of the text, i.e \(x^2 +4\),
       this is done by typing \tex{\(x^2 +4\)}. The other way is \emph{display}
       mode, where the expression is centered and larger, its written as
       \tex{\[x^2 +4\]}.  \[x^2 +4\]

     \item[Environemnts]
       For commands with a large number of parameters we can use the \emph{begin/end}
       construct. This is done by typing \tex{\begin{environment}[options]\end{environment}}
       where environment is the name of the environment and options are potential
       configurations.
   \end{description}

   Listing~\ref{lst:latex} shows a simple latex-fil, using the above
   \LaTeX{} commands.
   \begin{listing}[!h]
     \begin{minted}[tabsize=1]{latex}
				\documentclass{article}
				\usepackage{amsmath} % Package for maths
				\usepackage{graphicx} % Package for graphics

				\title{
					\large{Study Now} \\
					\Large{\LaTeX webinar} \\
				}

				\author{
					Benjamin Rotendahl --- Benjamin@Rotendahl.dk
				}

				\begin{document}
						\maketitle
						\section{Introduction}
							This documents gives you \(\dots\)
				\end{document}
   		\end{minted}
     \caption{Example of simple Latex document}\label{lst:latex}
   \end{listing}

 \subsection{Reference martial and problem solving}
   Learning \LaTeX{} is an iterative process, after having learned the basis you
   should start using it as much as possible, improving gradually. There are
   to many commands to learn all in one sitting. Once you start writing and
   run into something you don't know how to do yet, google it and remember it for
   next time. For instance if you want to know how to insert two figures side by
   side, a google search will most likely lead you to \emph{tex.stackexchange.com}
   which has several examples. For more advanced features and a comprehensive
   guide, checkout ``The Not So Short Introduction to
   \LaTeX\footnote{http://web.math.ku.dk/~holm/download/lshort.pdf}''.


   You will often, especially in the beginning, run into errors where the
   compiler fails to produce a pdf. The Compiler outputs a log where it tries
   to explain what went wrong in a helpful way. As an example if we mistype
   and write \tex{\newpge} in stead of \tex{\newpage}, the compiler will write
   the following in the log.
   \begin{verbatim}[h]
		.../study_now/latex/cheetsheet.tex:241: Undefined control sequence.
		l.241 ...ewpage} we write \newpge
	 \end{verbatim}
   The log tells us where the error happened and why. Once you learned to read
   the logs you will be able to fix errors quickly.



\section{Math in \LaTeX}
 Writing beautifully formatted math is one of the biggest strengths of \LaTeX{}.
 We start by telling the compiler to enter \emph{math mode}, it will then parse
 the following text as math. It's the difference between y+x and \(y+x\). Many
 commands only work in math mode, If you write \tex{\pi} in a non math mode
 the compiler will fail. Instead we write \tex{\(\pi\)}, resulting in \(\pi\).

 For larger math expressions we can use the \emph{display} mode, which is invoked
 by typing \tex{\[x+y\]} or by using the \tex{\equation} environment.
 Using the second approach makes it possible to create references to formulas.
 As an example the equation \eqref{dot} shows the dot product of two vectors.
 \begin{equation}\label{dot}
   \vec{a} \cdot \vec{b} = \sum_{i=1}^{n} a_i b_i
 \end{equation}

 The code to create the above expression is shown in listing~\ref{lst:dot}.
 \begin{listing}[!h]
   \begin{minted}[tabsize=1]{latex}
		\begin{equation}\label{dot} % Label to reference
			\vec{a} \cdot \vec{b} = \sum_{i=1}^{n} a_i b_i
		\end{equation}
	\end{minted}
   \caption{Equation for the dot product}\label{lst:dot}
 \end{listing}

 If we wish to have multiple steps in a derivation we can use the \tex{flalign}
 environment. This is a way to align equations in a derivation, we type each
 line use \tex{\\} to start a new line and set the \tex{&} symbol to specify
 the symbol of each line that should be centered.
 \begin{flalign}
   f(x, y)                       & = 3x^2y + y^2 \\
   \frac{\partial f}{\partial x} & = 6xy         \\
   \frac{\partial f}{\partial y} & = 3x^2 + 2y
 \end{flalign}
 The code to produce the above can be seen in ~\ref{lst:diff}. If you don't want
 line numbers on the right hand side type \tex{flalign*} instead of \tex{flalign}.
 \begin{listing}[!h]
   \begin{minted}[tabsize=1]{latex}
		\begin{flalign}
			f(x, y)                       & = 3x^2y + y^2 \\
			\frac{\partial f}{\partial x} & = 6xy         \\
			\frac{\partial f}{\partial y} & = 3x^2 + 2y
		\end{flalign}
 \end{minted}
   \caption{Example to create multi line math}\label{lst:diff}
 \end{listing}
 To write parenthesis in formulas we can type them normally, but this will not
 scale them according to their contents. To scale them type \tex{\left ( x^2 \right )}
 to use different types of parenthesis replace the symbol after \tex{\left, \right}.
 Listing~\ref{lst:par} shows the source for the following example
 \begin{flalign*}
   f(x) & = (\frac{1}{x})^2                              \\ % Non scaled
   f(x) & = \left ( \frac{1}{x} \right )^2               \\ % Scaled
   x    & \in \left [ \frac{1}{4},  \frac{1}{2} \right ] \\ % Scaled brackets
   x    & \in \left
   \{\frac{1}{n}, \frac{2}{n}, \frac{3}{n}, \dots, \frac{n}{n} \right \}
 \end{flalign*}

 \begin{listing}[!h]
   \begin{minted}[tabsize=1]{latex}
 \begin{flalign*}
   f(x) &= (\frac{1}{x})^2  \\ % % Non scaled
   f(x) &= \left ( \frac{1}{x} \right )^2  \\ % Scaled
   x  \in \left [ \frac{1}{4},  \frac{1}{2} \right ] \\ % scaled brackets
   x  \in \left \{\frac{1}{n}, \frac{2}{n}, \frac{3}{n}, \dots, \frac{n}{n} \right \}
 \end{flalign*}
\end{minted}
   \caption{Code to scale parenthesis}\label{lst:par}
 \end{listing}
 \subsection{Math operations and notation}
   The following is a list of the most common math notation.
   \begin{description}
     \item[Standard operations] Writing \(+, -\) is done normally, to do
       multiplication you can either use \tex{a \cdot b, a \times b},
       which becomes \(a \cdot b, a \times b\).
       \[
         1+1 -y \cdot x
       \]
     \item[Fractions] To write fractions we use the \tex{\frac{a}{b}} command.
       The content of the first argument is placed on the numerator and the second
       the denominator. Fractions can also contain fractions.
       \[
         \frac{a}{b} + \frac{a}{\frac{b}{c}}
       \]
     \item[Super/Sub-scripts] To write \(x^2, x_i\), one types
       \tex{\(x^2, x_i\)}. To include more than one symbol in a super/sub-script
       it has to be in a set of curly parenthesis, ie. \tex{\(x_{i+1}\)}, resulting
       in \(x_{i+1}\).

     \item[Sums and integrals] To write sums and integrals we write the name
       of the command followed by super and sub scripts.
       \tex{\sum_{i=1}^{n} a_i b_i} or \tex{\int_{i=1}^{n} a_i b_i}
       Resulting in:
       \[
         \sum_{i=1}^{n} a_i b_i + \int_{i=1}^{n} x
       \]
   \end{description}


 \subsection{Matrices \& Vectors}
   \LaTeX{} makes it easy to write vectors and matrices. It's done by using the
   environments: \\\tex{\pmatrix, \bmatrix, \bMatrix} which respectively creates
   a matrix with \( (, [, \{ \). In the environemnt you write the rows using
   \tex{&} to seperate the elements and \tex{\\} to seprate the rows. The
   following example used the code in listing~\ref{lst:matrix}. Note that
   the command \tex{\quad} was used to create spacing between the matrices and
   that the last matrix used the \tex{\dots, \vdots, \ddots} commands.
   \[
     \begin{pmatrix}
       1 & 2 & 3 \\
       4 & 5 & 6 \\
       7 & 8 & 9
     \end{pmatrix}
     ,\quad
     \begin{bmatrix}
       1 & 2 & 3 \\
       4 & 5 & 6 \\
       7 & 8 & 9
     \end{bmatrix}
     ,\quad
     \begin{Bmatrix}
       1 & 2 & 3 \\
       4 & 5 & 6 \\
       7 & 8 & 9
     \end{Bmatrix}
     ,\quad
     \begin{bmatrix}
       1      & \dots  & n      \\
       \vdots & \ddots & \vdots \\
       n      & \dots  & n
     \end{bmatrix}
   \]
   \begin{listing}[!h]
     \begin{minted}[tabsize=1]{latex}
			\[
				\begin{pmatrix}
					1 & 2 & 3 \\
					4 & 5 & 6 \\
					7 & 8 & 9
				\end{pmatrix}
				,\quad
				\begin{bmatrix}
					1 & 2 & 3 \\
					4 & 5 & 6 \\
					7 & 8 & 9
				\end{bmatrix}
				,\quad
				\begin{Bmatrix}
					1 & 2 & 3 \\
					4 & 5 & 6 \\
					7 & 8 & 9
				\end{Bmatrix},
				,\quad
				\begin{bmatrix}
					1      & \dots  & n      \\
					\vdots & \ddots & \vdots \\
					n      & \dots  & n
				\end{bmatrix}
			\]
 \end{minted}
     \caption{Examples of different types of matrices}\label{lst:matrix}
   \end{listing}


\section{Figures}
 To create figures with images and graphs we can use the \tex{figure}
 environment. A figure is \emph{floating} element, \LaTeX{} will place the
 figure in the document to minimize blank space.
 \begin{listing}[!h]
   \begin{minted}[tabsize=1]{latex}
			\begin{figure}[h]
				\centering\includegraphics[width=0.9\textwidth]{assets/compile.png}
				\caption{From source coude to pdf}\label{fig:compile}
			\end{figure}
		\end{minted}
   \caption{Example of inserting a figure}\label{lst:figure}
 \end{listing}
 We create a figure environment, adding the \tex{[h]} option to indicate that
 we prioritize having the figure close to the text. \emph{h} could be replaced by
 \tex{[t], [b]} which would place the figure at the top or bottom of the page
 respectively. The  \tex{\centering} command places the figure in the middle,
 and  \tex{\includegraphics} inserts the image, the \tex{\width=.9} specified
 the width of the image to \(90\%\) of the page width. The last part of the \tex{\includegraphics}
 command in curly parenthesis is the to the file file. Using \tex{\caption} and
 \tex{\label} we create a description of the figure and a label to reference.

 \subsection{TikZ}
   Using the tikz extension we can create our own graphs and figure directly from
   latex. TikZ is as powerful as it complex, learning it takes time and effort.
   Creating the graphs in another tool and exporting a png is also a viable option.
   Figure~\ref{fig:tikz} shows a graph created in TikZ and listing ~\ref{lst:tikz}
   shows how it was created.
   \begin{figure}
     \centering
     \begin{tikzpicture}
       \begin{axis}[xmax=9,ymax=9, samples=100]
         \addplot[blue, thick] (x,5*x^2+3*x);
         \addplot[red, thin] (x*x,x);
       \end{axis}
     \end{tikzpicture}
     \caption{Example of a tikz graph}\label{fik:tikz}
   \end{figure}

   \begin{listing}[!h]
     \begin{minted}[tabsize=1]{latex}
     \begin{figure}
      \centering
      \begin{tikzpicture}
       \begin{axis}[xmax=9,ymax=9, samples=50]
       \addplot[blue, thick] (x,5*x^2+3*x);
       \addplot[red, thin] (x*x,x);
     \end{axis}
   \end{tikzpicture}
  \caption{Example of a tikz graph}\label{tiks:graph}
 \end{figure}
   \end{minted}
     \caption{Example creating a figure in tikz}\label{lst:tikz}
   \end{listing}

\section{Lists}
 In latex we primerailliy use three different types of lists, in general their
 syntax is: \\\tex{\begin{list_type} \item thing 1 \item thing 2 \end{list_type}}.
 Lists can contain sub lists and handles bullets and enumerations automatically.
 The code to produce the following list can be seen in  listing~\ref{lst:list}
 \begin{description}
   \item[\tex{itemize}] A list with bullets.
     \begin{itemize}
       \item Some Bullet
       \item Some other Bullet
       \item A bullet point and a sub list  \begin{itemize}
               \item a sub bullet
               \item some other sub bullet
             \end{itemize}
     \end{itemize}
   \item[Enumerate] A list with numbered points, with a sub list where each sub
     point also has a unique id.
     \begin{enumerate}
       \item Some point
       \item Some other point
       \item Some other, other point with a sub list  \begin{enumerate}
               \item sub list 1
               \item sub list 2
             \end{enumerate}
     \end{enumerate}
   \item[\tex{description}] A list with a heading for each list item written as
     \tex{\item[header]}
     \begin{description}
       \item[First thing] first
       \item[Second thing] second thing
       \item[Third thing] list with sub headings \begin{description}
           \item[Fist sub thing] first sub thing
           \item[Second sub thing] second sub thing
         \end{description}
     \end{description}
 \end{description}
 \begin{listing}[!h]
   \begin{minted}[tabsize=1]{latex}
    \begin{description}
      \item[\tex{itemize}] A list with bullets.
        \begin{itemize}
          \item Some Bullet
          \item Some other Bullet
          \item A bullet point and a sub list  \begin{itemize}
                  \item a sub bullet
                  \item some other sub bullet
                \end{itemize}
        \end{itemize}
      \item[Enumerate] A list with numbered points, with a sub list where each sub
      point also has a unique id.
        \begin{enumerate}
          \item Some point
          \item Some other point
          \item Some other, other point with a sub list  \begin{enumerate}
                  \item sub list 1
                  \item sub list 2
                \end{enumerate}
        \end{enumerate}
      \item[\tex{description}] A list with a heading for each list item written as
     \tex{\item[header]}
        \begin{description}
          \item[First thing] first
          \item[Second thing] second thing
          \item[Third thing] list with sub headings \begin{description}
              \item[Fist sub thing] first sub thing
              \item[Second sub thing] second sub thing
            \end{description}
        \end{description}
    \end{description}
	 \end{minted}
   \caption{Different types of lists}\label{lst:list}
 \end{listing}



\section{Tables}
 Creating a table requires two environments \tex{\table, \tabular}, they function
 just as \tex{\figure}, \tex{\includegraphics}, one creating a float and the other the
 contents. Tabel~\ref{table:data} is the results of the code in  listing~\ref{lst:table}.
 As with the figure we add the \tex{[h]} option to have the table close to our
 text. The \tex{\tabular} environnement starts the actual table, the first
 pair of curly parenthesis indicate the number of elements in each row and where
 they justification. By typing \tex{\tabular}{|l|c r|}, we create three columns,
 the first column is left justified, the second is centered and the third is
 right justified. The pipe symbol \tex{|} creates a vertical bar between the
 columns. The elements of the table is written as a matrix using \tex{&} to
 separate elements and \tex{\\} to separate rows. The \tex{\hline} command can
 be added after a line to create a horizontal separator.
 \begin{table}[h]
   \centering
   \begin{tabular}{||c c | c c||}
     \hline
     column1 & column2 & column3 & column4 \\  \hline
     1       & 6       & 87837   & 787     \\
     2       & 7       & 78      & 5415    \\
     3       & 545     & 778     & 7507    \\
     4       & 545     & 18744   & 7560    \\
     5       & 88      & 788     & 6344    \\
     \hline
   \end{tabular}
   \caption{Example of table}\label{table:data}
 \end{table}
 \begin{listing}[!h]
   \begin{minted}[tabsize=1]{latex}
	  \begin{table}[h]
			\centering
			\begin{tabular}{||c c | c c||}
				\hline
        column1 & column2 & column3 & column4 \\  \hline
        1      & 6      & 87837  & 787    \\
        2      & 7      & 78     & 5415   \\
        3      & 545    & 778    & 7507   \\
        4      & 545    & 18744  & 7560   \\
        5      & 88     & 788    & 6344   \\
        \hline
			\end{tabular}
			\caption{Example of manual table}\label{table:data}
		\end{table}
	\end{minted}
   \caption{Example of manual table }\label{lst:table}
 \end{listing}
 \subsection{Automatic tables}
   If you have a file in csv format you can add the table automatically by
   including \tex{\usepackage{csvsimple}} in your preamble and typing:
   \begin{table}[h]
     \centering\csvautotabular{../assets/dices.csv}
     \caption{Automatic table from csv}\label{table:dices}
   \end{table}
   \begin{listing}[!h]
     \begin{minted}[tabsize=1]{latex}
     \begin{table}[h]
       \centering\csvautotabular{assets/dices.csv}
       \caption{Automatic table from csc}\label{table:dices}
     \end{table}
     \caption{Example of automatically generated table}\label{lst:dices}
		\end{minted}
     \caption{Example of automatically generated table}\label{lst:dices}
   \end{listing}


\section{Code listings}
 To include code listings in your document we can again either do as shown in
 the examples below:
 \begin{listing}[!h]
   \begin{lstlisting}[language=python, tabsize=1]
			def fib(x):
				if x == 1 or x == 2:
					return 1
				else:
					return fib(x-1) + fib(x-2)
			\end{lstlisting}
 \end{listing}
 Which was produced by:
 \begin{listing}[!h]
   \begin{minted}[tabsize=1]{latex}
		\begin{listing}[!h]
			\begin{lstlisting}[language=python, tabsize=1]
				 def fib(x):
					 if x == 1 or x == 2:
						 return 1
					 else:
						 return fib(x-1) + fib(x-2)
				 \end{lstlisting}
	\caption{Manual code listing }\label{lst:dices}
 \end{minted}
   \caption{Code to generate automatic listing}\label{lst:dices}
 \end{listing}

 To get prettier listings with syntax highlighting as in this example we must
 install the package \emph{minted}. Follow the guide on their page:
 \url{https://github.com/gpoore/minted}

\section{References}
 Adding labels to you contents allows you to easily reference them. Labels are
 written as \tex{\label{thing:name}} and can be added to almost any element,
 including but not limed to \\\tex{\section, \subsection, \figure, \table, \equation}.
   Once a label has been made we can refer to it by: \tex{\ref{thing:name}}.
   Labels are used for internal references in the document, to refer to other
   articles and papers we use the extension \emph{bibtex}.
   These are written in a file named along the lines \emph{references.bib} and
   have contents similar to listing listing~\ref{lst:bib}
   \begin{listing}[!h]
     \begin{minted}[tabsize=1]{bibtex}
		@article{attention,
		author    = {Vaswani, Ashish and Shazeer, Noam and Parmar, Niki and Uszkoreit},
		eprint    = {1706.03762},
		journal   = {arXiv},
		keywords  = {Language Transformer},
		rating    = {5},
		title     = {{Attention Is All You Need}},
		year      = {2017}
	}

	@misc{latexPackage,
		howpublished = {\url{
			https://marketplace.visualstudio.com/items?itemName=James-Yu.latex-workshop
		}},
		title        = {VS Code Latex Package}
	}
	@misc{texLive,
		howpublished = {\url{https://www.latex-project.org/get/}},
		title        = {The LaTex Project}
	}
	@misc{vscode,
		howpublished = {\url{https://code.visualstudio.com}},
		title        = {Microsoft Visual Studio Code}
	}
\end{minted}
     \caption{Code to generate automatic listing}\label{lst:bib}
   \end{listing}
   One the file has been created we add the following to the bottom of our document to
   generate the references.
   \begin{listing}[!h]
     \begin{minted}[tabsize=1]{latex}
		\bibliography{assets/references}{}
		\bibliographystyle{plain}
\end{minted}
     \caption{Code to generate automatic listing}\label{lst:dices}
   \end{listing}
   To reference an article we type:
   \tex{\cite{attention}} resulting in
   \emph{Attention Is All You Need}\cite{attention}



   \newpage
   \appendix
   \bibliography{../assets/references}{}
   \bibliographystyle{plain}


\end{document}
